\documentclass[11pt]{article}

\usepackage{graphicx} %,grffile}
%\usepackage[small]{caption} %small font captions
%\usepackage{parskip} %for no indent and more space in between paragraphs
\usepackage[letterpaper,bottom=1in,top=1in,right=1.25in,left=1.25in,includemp=FALSE,includeheadfoot=FALSE,headheight=0pt]{geometry}
\usepackage[natbib=true,backend=biber,style=chicago-authordate]{biblatex}
\addbibresource{refs.bib}
\usepackage{float}
\usepackage{amssymb,amsmath}
%\usepackage{multirow}
%\usepackage{multicol}

%  https://tex.stackexchange.com/questions/49788/hyperref-url-long-url-with-dashes-wont-break
%\PassOptionsToPackage{hyphens}{url}
\usepackage[colorlinks  = true,
            linkcolor   = blue,
            urlcolor    = blue,
            citecolor   = blue,
            anchorcolor = white,
	    unicode = true ]{hyperref}
%\usepackage[hyphens]{url}
%\urlstyle{same}  % don't use monospace font for urls

%\usepackage[T1]{fontenc}
%\usepackage[utf8]{inputenc}
%\usepackage{textcomp} % provides euro and other symbols
% use upquote if available, for straight quotes in verbatim environments
\usepackage{upquote}
% use microtype if available
%\IfFileExists{microtype.sty}{%
%	\usepackage[]{microtype}
%	\UseMicrotypeSet[protrusion]{basicmath} % disable protrusion for tt fonts
%}{}
\usepackage{parskip}
\usepackage{xcolor}
\usepackage{soul}
\usepackage{longtable}
\usepackage{booktabs}
%\usepackage{pdflscape}
\usepackage{lscape}
%\usepackage{rotating}
% Fix footnotes in tables (requires footnote package)
%\IfFileExists{footnote.sty}{\usepackage{footnote}\makesavenoteenv{longtable}}{}
%\makeatletter
%\def\maxwidth{\ifdim\Gin@nat@width>\linewidth\linewidth\else\Gin@nat@width\fi}
%\def\maxheight{\ifdim\Gin@nat@height>\textheight\textheight\else\Gin@nat@height\fi}
%\makeatother
% Scale images if necessary, so that they will not overflow the page
% margins by default, and it is still possible to overwrite the defaults
% using explicit options in \includegraphics[width, height, ...]{}
%\setkeys{Gin}{width=\maxwidth,height=\maxheight,keepaspectratio}
% \setlength{\emergencystretch}{3em}  % prevent overfull lines
% \providecommand{\tightlist}{%
%   \setlength{\itemsep}{0pt}\setlength{\parskip}{0pt}}
% % \setcounter{secnumdepth}{0}
% % Redefines (sub)paragraphs to behave more like sections
% \ifx\paragraph\undefined\else
% \let\oldparagraph\paragraph
% \renewcommand{\paragraph}[1]{\oldparagraph{#1}\mbox{}}
% \fi
% \ifx\subparagraph\undefined\else
% \let\oldsubparagraph\subparagraph
% \renewcommand{\subparagraph}[1]{\oldsubparagraph{#1}\mbox{}}
% \fi

% set default figure placement to htbp
%\makeatletter
%\def\fps@figure{htbp}
%\makeatother

\author{Rosario Queirolo\thanks{Universidad Católica del Uruguay (\url{mailto:rosario.queirolo@ucu.edu.uy})}
	\and Jake Bowers\thanks{University of Illinois @ Urbana-Champaign}
	\and Eliana Álvarez\thanks{Universidad Católica del Uruguay}
	\and Lorena Repetto\thanks{Universidad Católica del Uruguay}}

\title{The Impact of Marijuana Sale at Pharmacies on Crime Victimization and Insecurity Perceptions.\thanks{This research was supported by Open Society Foundations funds OR2017-35193 and OR2016-27307, and the National Agency for Research and Innovation (ANII) between 2017 and 2019.}}

\graphicspath{{.}{Analysis/}{media/}}

\begin{document}

\maketitle


\begin{abstract}

TO BE REVISED AFTER FINISHING Do drugs' legalization have an impact on crime and public security? This paper presents the results of an observational design that evaluates the impact that recreational marijuana sale at pharmacies had on insecurity perceptions and crime victimization in Uruguay. Existing evidence about the impact of legal marijuana dispensaries on crime is not conclusive, and mainly based on the United States experience where national crime rates have been decreasing. On the contrary, Uruguay's crime rates have been consistently increasing in the last five years. Less is known about how legalization policies affect people's perceived public insecurity. We performed two surveys, one before and one after the implementation of the selling at pharmacies started. The sample is composed by 1.298 neighbors of 64 pharmacies in all the territory, 10 or more neighbors per pharmacy in each round. These outcomes included country’s insecurity perception, neighborhood’s insecurity perception, crime victimization in the last twelve months, perceived impact of the legalization on public insecurity, perceived impact on drug trafficking, reported existence of ``bocas'', insertion in the neighborhood and social disorder. \hl{Data is analyzed by differences in differences (DD) to estimate the causal effect before and after the implementation of the marijuana regulation. We found no significant impact of closeness to dispensaries on crime victimization}. Neither on public insecurity. This result add to part of the research done on the impact of marijuana legalization in several US states. We do find an impact on the perception of drug trafficking: neighbors of marijuana selling pharmacies consider that the legalization had a positive impact on traffic. While we cannot pin down the mechanisms behind this effect, we believe that seeing users buying legal marijuana might lead citizens to think that illegal deals have diminished.

\end{abstract}
%\begin{keywords}
% Crime, Legal Marijuana dispensaries, Natural experiment, insecurity perception
%\end{keywords}

\section[]{Laws, Crime, Public Insecurity, and the Sale of Marijuana in Neighborhoods in Uruguay}

In 2013, Uruguay became the first country in the world to fully regulate the production, distribution and commercialization of marijuana. The law had three main objectives: decriminalize users, reduce public insecurity and drug-trafficking related violence, and increase public health through education, risk reduction and prevention campaigns, now made easy to implement because of their reference to a legal substance \citep{arraras2014inventando, pardo2014cannabis, queirolo2019uruguay}. The sale of marijuana at pharmacies was implemented four years after the regulation approval, on July 19th., 2017. Much of the Uruguayan's were against the legalization in general, and the sale at pharmacies in particular. Only 16 out of more than 1000 pharmacies in the country registered to sell marijuana. This paper focuses on the impact that living close to a marijuana dispensary has on crime victimization and public insecurity perception, adding to a growing literature that analyze the impact of marijuana legalization on crime.  

In the context of pharmacies and neighborhoods in Uruguay, a pharmacy-level change is a neighborhood level change. Would living close to a marijuana selling pharmacy have an impact on public insecurity perceptions and crime victimization? To learn about the dynamics of this process and to document them, our research team fielded in-person interviews of roughly 10 neighbors of each of the 16 registered pharmacies as well as with 10 neighbors living near 42 pharmacies chosen to be similar to the 16 registered pharmacies but which had not registered to sell marijuana during the month before sale of marijuana was made legal. We also re-surveyed those neighborhoods one year later, in August of 2018.

\paragraph{Why document the effect of the sale of marijuana on crime victimization and public insecurity?}\\ 

We hope to contribute to the public policy discussion in Uruguay and elsewhere on whether, on balance, marijuana legalization may or may not have an impact on crime rates. Research has found mixed results about this relationship in general. Braakmann and Jones (\citeyear{braakmann2014cannabis}) for declassification cannabis laws in England and Wales, and Chu and Townsend (\citeyear{chu2019joint}) for US states with medical marijuana laws, find no significant relationship between these laws and crime. Hunt, Pacula and Weinberger (\citeyear{hunt2018high}) find no relationship between Californian counties with laws that legally permit dispensaries and reported violent crimes. At the same time, in Colorado, two studies find contradictory evidence. Freisthler et al.(\citeyear{freisthler2016micro}) show that the presence of marijuana dispensaries affects crime not where stores are located but in adjacent areas. In the contrary direction, Brinkman and Mok-Lamme (\citeyear{brinkman2019not}) finds that an additional dispensary has a significant impact on reducing crime on the neighborhood, particularly on non violent crimes, with no spillover effects to adjacent areas.  Other evidence from  quasi-experimental research has shown a decrease in some types of crimes in the areas around marijuana dispensaries \citep{dragone2019crime, gavrilova2014legal, Indigo:2016}.

Our research differs to the previously reviewed in three aspects. First, most of the articles published look at the relationship between cannabis laws and crime using administrative data from national agencies, mainly through crime reports. In this article, we followed a different path. In addition to measure crime victimization using individual self reports, we also measure people's perceptions: public insecurity perceptions and perceived impacts of marijuana legalization. The rationale to do so is that cannabis laws might have an impact on public insecurity perceptions in addition to on crime. We implemented a national survey designed specifically to measure these outcomes. Evidence shows that simple perceptions of demographics of neighborhoods often diverge from census reported figures \citep{wong2012bringing}, and fear of crime does not necessarily reflect crime prevalence or the probability of crime. As a result, understanding perceptions is as important as understanding actual crime figures. 

Second, main evidence on the relationship between crimes and legalization comes from states in USA, where it has been a sustained downward trend in national crime rates \citep{gramlich5facts, james2018recent}. The situation is exactly the opposite in Uruguay, where national crime rates show a significant increase in last years \citep{del2018obstacles, paternain2008panorama, aboal2007crimen}. It is highly probable that these contextual differences might alter the way in which cannabis laws impact crime and public insecurity.

Third, most of the published articles asking the question of marijuana legalization impact on crime, use differences in differences analysis. In this article we try to approach an observational design to an experimental one using matching. 

These differences among the research already performed and ours, influence the expectations about the results. Regarding people's perceptions, the fact that a person's idea about the probability of crime may not be the same of a local police department or criminologist, means that we do not have strong priors about how a pharmacy selling marijuana may change such perceptions. If perceptions are malleable, then perhaps a new pharmacy might lead to quick change; if, however, people ignore their surroundings and/or only perceive which fits in their previous beliefs, then perhaps the pharmacies may have no effects. In addition, given the national crime trends in the country, we should not expect that legalization would have the same effect than in contexts where crime has been diminishing \citep{eisner2016achieving}.

The rest of the paper is structured as follow. Section two describes the context, in particularly, the way in which the public security argument was introduced in the implementation of marijuana selling at pharmacies and how the selling currently works. The third section summarizes the evidence about the relationship between marijuana legalization and crime, describes the main outcomes we are looking at, and states our main hypotheses. Fourth, we explain our research design. In the fifth section, results are presented; and in the  sixth section we discuss how well or bad our design deals with unobserved confounds. Finally, we end up with a discussion section.

\section{Marijuana Retail at Pharmacies: A Controversial Measure}
As we mentioned, only 16 pharmacies decided to start selling marijuana on July 19th. 2017. The reasons given by some pharmacists for not entering the system varied. Some of them rejected the idea of selling a recreational drug; other pharmacy owners claimed uncertainty about the functioning of the system. Still others worried about the economic profitability of the initiative or about the potential threat to their security and reprisals from dealers \citep{boidi2016}. Currently, the Institute of Regulation and Control of Cannabis (IRCCA) reports 17 pharmacies active in the dispensary system, located in only 10 of the 19 departments of the country.

Main opposition to the idea of selling marijuana at pharmacies came from associations of the pharmaceutical industry. Even before its implementation, the main professional organizations of pharmacists and owners of pharmacies came out against the measure.\footnote{These organizations included the professional organization of pharmaceutical chemists (the Association of Chemistry and Pharmacy of Uruguay (AFQU)), and the professional organization of owners of pharmacies in Montevideo (the Pharmaceutical Chamber (CF)) and the Association of Pharmacies of Inland (AFI).} For example, the Association of Chemistry and Pharmacy of Uruguay publicly stated that pharmacies exist to provide healthcare and the sale of a recreational drug was not consistent with the primary purpose of pharmacies. The Association of Pharmacies of Inland voiced concerns about public security: ``There is a lot of fear of becoming highly attractive for thieves, especially on peripheral pharmacists''\footnote{Diario El Pais.``Marijuana: pharmacies fear assaults''. Published on May 9, 2016. In: \url{https://www.elpais.com.uy/informacion/marihuana-farmacias-temen-asaltos.html} Original quote in spanish: ``Hay mucho temor de que sean posibles llamadores de asaltos, especialmente farmacias periférica''}. In the same newspaper article, the professional organization of owners of pharmacies in Montevideo expressed the same concerns: ``We all know that pharmacies are highly precious for criminals. Pharmacists had been killed, and in poor neighborhoods on the edges of the city, people steal to buy a joint. And now, marijuana will be in the pharmacies''\footnote{Diario El Pais. ``Marijuana: pharmacies fear assaults''. Published on May 9, 2016. In: \url{https://www.elpais.com.uy/informacion/marihuana-farmacias-temen-asaltos.html} Original quote in Spanish:  ``Todos sabemos que la farmacia barrial es un botín muy preciado para los delincuentes. Hemos tenido farmacéuticos asesinados y en barrios periféricos te roban para comprar un porro y ahora la marihuana va a estar en la propia farmacia''}.

Security concerns were also present at the time of the implementation of marijuana legalization in the United States. Even though states that legalized marijuana did it with a strong popular support, in some sectors the legalization wave produced concerns about the crime impact of this policy, not just within the state but in the neighbor ones. For instance, in October 2016, the Denver District Attorney wrote a letter warning Californian citizens that have voted for Proposition 64 (Adult Use of Marijuana Act) about the effect of recreational marijuana legalization on the raising of crime in Denver and Colorado \citep{dragone2019crime}.

\section{The Legalization Effect in a Context of Increasing Crime}
The academic discussion about the relationship between drugs and crime has not reach consensus \citep{white2000dynamics}. Specially, marijuana occupies a particular place in this discussion. Most studies have concluded that marijuana consumption tends to inhibit violent behavior in the short term, disqualifying the psychopharmacological argument \citep{white2000dynamics} that states that crime occurs as a consequence of drug consumption \citep{goldstein1985drugs}.\footnote{Although a review by the National Research Council concluded that long-term cannabis use can affect the nervous system prompting violence behavior \citep{national1994understanding}.} By the same reasoning, marijuana use rarely generates economic violence as it does not produce the compulsive need to generate income to fund consumption since marijuana is relatively cheap \citep{caulkins2016marijuana}. Finally, given the low profitability of marijuana markets in comparison with other drugs, such as cocaine and heroin, observers believe that marijuana use does not play a significant part in the systemic type of violence \citep{pacula2003marijuana, caulkins2015considering} which relates to drug-trafficking and distribution \citep{goldstein1985drugs}. Thus, the legalization of marijuana should not have large short term effects on violence unless the removal of marijuana from the portfolios of illegal dealers somehow increases violence and competition in the cocaine and other drug markets.

Despite these reasons, recent evidence for the US states that have legalized marijuana shows that these laws have an effect on the reduction of violent crimes and property crimes \citep{dragone2019crime, Indigo:2016, gavrilova2014legal, huber2016cannabis, brinkman2017not}. The causal mechanisms behind these relationships are still unclear. Based on a review of the existing evidence, Dragone et al.(\citeyear{dragone2019crime}) suggest that they might be: 1) that the psychotropic effect of consuming marijuana is sedative, which decreases the chances of getting involved in violent crimes \citep{no2001health, green2003being}; or 2) that users of alcohol and other drugs are substituting their consumption with marijuana which is a less violent drug \citep{anderson2014legalization, kelly2014policing} ; or 3) that the decrease in crime rate is an effect of the reallocation of police efforts which after the legalization does not need to prosecute marijuana consumption \citep{adda2014crime}; or 4) that by withdrawing the sale of marijuana from unsafe places and placing it in legal and secure contexts, criminal gangs, intermediaries and dealers become weaker \citep{becker2013have}. 

However, this evidence comes from contexts with opposite crime trends to those of Uruguay. In United States crime rates have been showing a drop in time \citep{gramlich5facts, james2018recent}, so maybe these findings might be related to the that fact. On the contrary, in Uruguay homicides, robberies and assaults rates have been increasing at least since 2012, as we can see on Figure~\ref{fig:homrap19802017}. Considering this difference, our study can show how legalization processes might have different results depending on national circumstances.

\begin{figure}[H]
	\centering
	\caption{Number of homicides and assaults in Uruguay. 1980 -- 2018} 
	\label{fig:homrap19802017}
	\includegraphics[width=.7\textwidth]{evo_delitos.pdf}
	\\
    \scriptsize{Source: Observatorio de Violencia y Criminalidad del Ministerio del Interior. \\ For years 1993 and 1994 annual data on homicides was not available.} 
\end{figure}

Not only crime has been increasing in Uruguay, also public concern towards insecurity. Public opinion studies show that public insecurity is one of the most important problems for Uruguayans: 61\% of Uruguayans considered ''Insecurity'' as the main problem of the country, even though 51\% of them never experienced a crime \footnote{CIFRA (July 2018) \url{http://www.cifra.com.uy/index.php/2018/08/20/inseguridad-el-problema-mas-grave-que-afecta-mas-a-jovenes-y-mujeres/}}. This proportion has been increasing since 2007, even before the crime rate peaked.

\subsection{Rationale}
Why might sale of marijuana in local pharmacies change crime victimization, fear of crime and insecurity perceptions? One possible avenue is that legalization of marijuana might increase marijuana use, and increased consumption might increase crime (\cite{pacula2003marijuana}) (which would in turn be perceived or directly experienced by neighbors leading to a higher public perception of insecurity). Other related way is that pharmacies became more attractive for thieves because they sell marijuana, and because they do it in cash (\cite{hunt2018high}). This is the argument that many pharmacies owners claimed for not adhering to the selling.

One of the motivation of the passage of Law 19.172 (the law legalizing marijuana in Uruguay) was to reduce drug related crimes by removing marijuana users from the illegal market and undermining the economic power of drug trafficking. Following government's goal, we could expect that the sale of marijuana in pharmacies will reduce crime and increase public security because dealers and illegal selling points ``bocas'' will move away from the neighborhood.

It has been well established that personal crime experiences -victimization- are one of the most important predictors of insecurity perceptions \citep{cruz2009public}. This means that people who were victimized by crime tend to feel more insecure than those who did not, even if they live in the same area. But, as we mentioned, fear of crime does not necessarily correspond to actual crime rates and their trends \citep{wong2012bringing}; other elements influence peoples' perceptions regarding public safety and the possibility of experience a crime. Neighborhood's social daily dynamics, infrastructure and general environment, also play a role in insecurity perceptions. Social conditions of neighborhoods can influence how people feels in terms of safety \citep{cruz2009public, brunton2011neighborhoods, valera2014perceived}. Studies have named these conditions as \textit{social disorder} or \textit{incivilities}. Situations like loitering, people drinking on the streets, gangs' presence, street harassment, etc. \citep{bennett1994determinants, valera2014perceived} are usually used to illustrate signs of social disorder. As more common or frequent these situations become, more unsafe people feel, generating a direct relation between social disorder and fear of crime.

People using and/or buying drugs in public spaces can also be considered a sign of social disorder. Social stereotype of drug users usually conceived them as unpredictable, and therefore, a threat to their neighbors \citep{bennett1994determinants}. Social contact produced by sale at pharmacies between marijuana users and non-users neighbors, either at the  moment of the purchase or because of the long lines at the pharmacies' entrance \footnote{Agencia EFE. ``Uruguayans make long lines to buy marijuana at pharmacies". Published on July 19, 2017. In: \url{https://www.efe.com/efe/america/sociedad/uruguayos-hacen-largas-colas-para-comprar-marihuana-en-las-farmacias/20000013-3330652} Original quote in spanish:"Uruguayos hacen largas colas para comprar marihuana en las farmacias''}, might be having an effect on the perceived social disorder of the neighborhood where the selling pharmacy is located, and by this, influencing neighbors' fear of crime.

Do we expect that neighborhood pharmacies selling marijuana should change local direct experience with crime and/or public perceptions of safety? We interpret the research based in the US states as arguing against increasing criminal victimization and pointing out a diminishing effect. Yet crime is rising in Uruguay, and concerns about safety have been increasing in Uruguay as well, even before crime began to rise. So, we will take some care to ensure that our comparisons of survey respondents and neighborhoods below do not confuse differences in public security perceptions caused by overall trends with differences that we can attribute to the introduction of marijuana sales at local pharmacies.

\subsection{Outcome Variables}
The main outcomes of this paper are crime victimization and insecurity perceptions. We measure crime by asking neighbors if they have experienced a crime in the last 12 months. In order to measure public insecurity perceptions, we use two variables: public insecurity perception in the neighborhood and public insecurity perception in the country. Among these two variables, we expect that marijuana selling pharmacies would have a higher effect on how secure or insecure citizens feel in the neighborhood rather than in the country.

To better understand the impact on public insecurity perceptions, we include three other outcome variables that we argue are related with public insecurity perceptions and fear of crime. First, we use an index that measure social disorder in the neighborhood composed by three variables: presence of young people loitering, presence of drunk or stoned people in the streets, and presence of people arguing with each other. This information was registered by the interviewer looking at the street where the respondent lives, after the face to face interviews were finished. Second, we include a variable that measures if neighbors know about the existence of illegal selling points (``bocas'') in the neighborhood. Considering that ``bocas'' bring illegal activities to the neighborhood which might imply violence, knowing that there is a ``boca'' close by  would be related with citizens insecurity perceptions. Third, we build an index that measure citizens insertion in the neighborhood comprised by four dimensions: use of services (education and health) in the neighborhood, contact among neighbors (chat and/or meet for collective action activities), perform recreational activities in the neighborhood, and shopping in the neighborhood. Neighborhood insertion might be also related to public insecurity perception and fear of crime.

Finally, we considered two outcomes concerning the perceived impacts of the legalization on public safety and drug-trafficking, which are among the main goals of the law. Having a marijuana selling pharmacy in the neighborhood might have an effect on how citizens evaluate the results of the legalization. Neighbors might see marijuana users buying at a pharmacy and conclude that legalization is hindering the illegal market. More details on how these variables are measured and indexes constructed are in Appendix B.

\subsection{Primary Hypotheses}
We list the primary hypotheses of the study below.

\begin{itemize}
\item [\textbf{H1}] \textit{The sale of marijuana at pharmacies will not have significant effects on crime victimization of pharmacies' neighbors}

\item [\textbf{H2}] \textit{The sale of marijuana at pharmacies will not have significant effects on neighbors' insecurity perceptions}

\item [\textbf{H3}] \textit{The sale of marijuana at pharmacies will push ``bocas'' outside the neighborhood}

\item [\textbf{H4}] \textit{The sale of marijuana at pharmacies will increase social disorder in the neighborhood}

\item [\textbf{H5}] \textit{The sale of marijuana at pharmacies will diminish citizens' insertion in the neighborhood}

\item [\textbf{H6}] \textit{The sale of marijuana at pharmacies will not change citizens' evaluations about the impact of marijuana legalization on public security}

\item [\textbf{H7}] \textit{The sale of marijuana at pharmacies will improve citizens' evaluations about the impact of marijuana legalization on drug trafficking}

\end{itemize}

\section{Research Design and Methods: Comparing pharmacies and Neighbors' Attitudes}
\subsection{Data Collection: Pharmacies and Neighbors Survey}
The data used in this analysis come from face-to-face surveys that aimed to help us measure the crime victimization experienced by those living near pharmacies and their attitudes towards the law and public insecurity. None of these surveys were intended to be representative of the Uruguayan population. We conducted two rounds of surveys with neighbors of selling and non selling pharmacies. In the first survey round, fieldwork started on June 17, 2017 and finished in July 1, 2017 before the sales started (July 19, 2017). The sample contains 600 neighbors of 60 pharmacies, 10 neighbors per pharmacy. The list of pharmacies joining the dispensation was confidential until the sale started, so this initial group of neighbors did not know if their closest pharmacy was going to sell. 

The second round of the survey was carried out approximately one year later, between August 9 and September 30, 2018. In total, we have conducted 1,298 interviews across the country in the two rounds. Respondents were selected among people over 18 years old that lived in the sampled household and were present at the moment of the interview. The most recent birthday selection process was used to choose among households residents.

\subsection{Selling and Non-selling Pharmacies}
In addition to neighbors, we also interviewed one representative of each pharmacy (sellers and non-sellers) in both rounds. In total, we conducted 119 surveys with representatives (59 in 2017 and 60 en 2018). Because marijuana selling was not randomly assigned by the government to pharmacies, and each pharmacy decided to adhere or not to the selling, those that sell and those that do not sell might be different. Data from the pharmacies' representatives survey allowed us to characterize selling and non-selling pharmacies, and analyze if there were any initial differences between them that could explain the differences on outcomes.

Selling and non-selling pharmacies are not different in terms of their general characteristics. The only relevant distinction is that among non-selling there are stores that are part of a pharmacy's chain. In terms of infrastructure both selling and non-selling pharmacies are quite similar, and they are also similar regarding security measures, insecurity perceptions and crime victimization. The later goes against the idea that pharmacies that sell marijuana did so because they are better equipped in terms of security protection, have suffered less crimes and feel more secure in their neighborhoods. There are not significant differences either in their perceptions about marijuana and its users. 

The most important distinction among sellers and non-sellers is concerning opinions towards the public policy. Owners and representatives of pharmacies that decided to sell are consistently more in favor of the regulation and anticipate it would have more positive impacts. Balance tests (see Table \ref{tab:pharmaciesperceptions} in Appendix A) show statistically significant differences between sellers and non-sellers among almost all outcomes related with policy opinions. The only exception is the perceived impact on drug trafficking in which there are no significant differences. In addition sellers are significantly more against the idea that marijuana is a gateway drug than non sellers. 

\subsubsection{Changes in Selling Pharmacies}
Our initial sample contained 42 non-selling pharmacies and 18 selling ones. Figure \ref{fig:ctrltrtpharms_1} shows the geographical distribution of the pharmacies in our study. During the first year of policy implementation, some changes in the pharmacies selling group occurred. First, two pharmacies that initially were registered to sell, never actually did it, this reduced our selling group to 16.  Second, six pharmacies that had entered the system abandoned it due to the prohibition imposed by US banks that forbid marijuana selling pharmacies to operate with bank accounts.\footnote{For information about the problem with the banking system see: \url{https://www.elobservador.com.uy/nota/brou-cierra-cuentas-vinculadas-con-marihuana-y-mas-farmacias-evaluan-dejar-de-venderla-20178175004}.} And third, because pharmacy registration remained open, 4 new pharmacies entered the system. 

These situations produced that, in the second round of the survey collected in 2018, we had different pharmacies status. On one hand, pharmacies that never sold and never registered to do it (n=42), named as ``control'' or ``comparison'' pharmacies. Also, two pharmacies that never sold but initially were willing to do it (n=2) named as ``placebo''. Plus, pharmacies that sold marijuana during the entire time between the first and second round (n=10) named as ``wholetime'', pharmacies that dropped out from 2017 to 2018 (n=6) that we call ``dropouts'', and pharmacies that joined the system at some point between the 2017 round and 2018 round (n=4) that we call ``newcomers'' (see Table \ref{tab:pharmsalestatus}).

\begin{figure}[htpb!]
    \centering
    \label{fig:ctrltrtpharms_1}
    \caption{Pharmacies registered to sell marijuana and comparison pharmacies}
    \includegraphics[width=6.5in,height=3.4851in]{./media/country.png}
\footnotesize{Note: green symbols show pharmacies registered to sell marijuana as of June 2017. \\ Blue symbols are comparison pharmacies.}
\end{figure}

The comparison group was selected following four criteria. The first criteria was``same  geographical unit'':  pharmacies in ``departamento'' without at least one selling pharmacy were deleted from the sampling frame. The second criteria was ``population density'':  pharmacies in rural areas were deleted from the sample because none of the selling pharmacies is located in rural areas. The third criteria was``criminality rate'': pharmacies in neighborhoods (for Montevideo) or cities where homicides reports, assaults reports and robberies reports are too high or too low  in comparison with the neighborhoods were selling pharmacies are situated, were discarded. The fourth criteria is ``distance'': pharmacies in the control group should be at least 6 blocks away from any treatment pharmacy. After eliminating the pharmacies that did not pass the four criteria, we randomly selected 42 control pharmacies among the 1000 pharmacies that exist in the country \footnote{We used official sociodemographic information about the neighbourhoods/localities/``departamentos''. See Table \ref{tab:phlongtable} in Appendix A for more details on the pharmacies characteristics}.

\begin{small}
\begin{table}[htbp!]
\scriptsize
    \centering
     \caption{Status of  marijuana selling pharmacies}
    \label{tab:pharmsalestatus}
    \begin{tabular}{@{}lllcccccc@{}}
\textbf{Pharmacy}	&	\textbf{Neighborhood}	&	\textbf{Dpt.}	&	\textbf{Round 1}	&	\textbf{Round 2}	&	\textbf{Entrance}	&	\textbf{Drop out}	&	\textbf{Type}	\\
\midrule
Antartida	 	&	 	Centro	 	&	 	Montevideo	 	&	 	Yes	 	&	 	Yes	 	&	 	7/19/17	 	&	 		 	&	 	wholetime	\\
 Caceres 	 	&	 	Pocitos	 	&	 	Montevideo	 	&	 	Yes	 	&	 	Yes	 	&	 	7/19/17	 	&	 		 	&	 	wholetime	\\
 Tapie	 	&	 	Ciudad vieja	 	&	 	Montevideo	 	&	 	Yes	 	&	 	Yes	 	&	 	7/19/17	 	&	 		 	&	 	wholetime	\\
 Las toscas	 	&	 	Las toscas	 	&	 	Canelones	 	&	 	Yes	 	&	 	Yes	 	&	 	7/19/17	 	&	 		 	&	 	wholetime	\\
 Nueva Brun	 	&	 	Trinidad	 	&	 	Flores	 	&	 	Yes	 	&	 	Yes	 	&	 	7/19/17	 	&	 		 	&	 	wholetime	\\
 Gortari	 	&	 	Centro	 	&	 	Lavalleja	 	&	 	Yes	 	&	 	Yes	 	&	 	7/19/17	 	&	 		 	&	 	wholetime	\\
 La Cabina	 	&	 	Las Flores	 	&	 	Maldonado	 	&	 	Yes	 	&	 	Yes	 	&	 	7/19/17	 	&	 		 	&	 	wholetime	\\
 Termal guaviyu	 	&	 	Termas	 	&	 	Paysandú	 	&	 	Yes	 	&	 	Yes	 	&	 	7/19/17	 	&	 		 	&	 	wholetime	\\
 Albisu Termal	 	&	 	Pasaje Comercial	 	&	 	Salto	 	&	 	Yes	 	&	 	Yes	 	&	 	7/19/17	 	&	 		 	&	 	wholetime	\\
 Bengoechea	 	&	 	Centro	 	&	 	Tacurembó	 	&	 	Yes	 	&	 	Yes	 	&	 	7/19/17	 	&	 		 	&	 	wholetime	\\
 Miguel	 	&	 	Canelones centro	 	&	 	Canelones	 	&	 	Yes	 	&	 	Yes	 	&	 	7/19/17	 	&	 	10/2/17	 	&	 	dropouts	\\
 Carmelo	 	&	 	Carmelo	 	&	 	Colonia	 	&	 	Yes	 	&	 	Yes	 	&	 	7/19/17	 	&	 	8/25/17	 	&	 	dropouts	\\
 Pitagoras	 	&	 	Malvin norte	 	&	 	Montevideo	 	&	 	Yes	 	&	 	Yes	 	&	 	7/19/17	 	&	 	8/9/17	 	&	 	dropouts	\\
 Saga	 	&	 	Centro	 	&	 	Artigas	 	&	 	Yes	 	&	 	Yes	 	&	 	7/19/17	 	&	 	9/6/17	 	&	 	dropouts	\\
 Medicci	 	&	 	Paysandú	 	&	 	Paysandú	 	&	 	Yes	 	&	 	Yes	 	&	 	7/19/17	 	&	 	9/1/17	 	&	 	dropouts	\\
 Bidegain	 	&	 	Libertad	 	&	 	San José	 	&	 	Yes	 	&	 	Yes	 	&	 	7/19/17	 	&	 	9/1/17	 	&	 	dropouts	\\
 Camaño	 	&	 	Pocitos	 	&	 	Montevideo	 	&	 	No	 	&	 	Yes	 	&	 	9/11/17	 	&	 		 	&	 	newcomers	\\
 Silleda	 	&	 	Brazo oriental	 	&	 	Montevideo	 	&	 	No	 	&	 	Yes	 	&	 	9/11/17	 	&	 		 	&	 	newcomers\\
 Constitución Sur	 	&	 	Flor de Maroñas	 	&	 	Montevideo	 	&	 	No	 	&	 	Yes	 	&	 	20/4/18	 	&	 		 	&	 	newcomers	\\
 Lilen	 	&	 	Punta Carretas	 	&	 	Montevideo	 	&	 	No	 	&	 	Yes	 	&	 	17/5/18	 	&	 		 	&	 	newcomers	\\
\bottomrule
\end{tabular}
\end{table}
\end{small}

\subsection{Stratification for Fair Comparison}
To learn about whether sale of marijuana in pharmacies influences crime victimization, attitudes and perceptions of neighbors, we compare the survey responses of neighbors living near pharmacies selling marijuana to the responses of neighbors who did not live near such pharmacies. The problem arising from a strategy like this, is that neighborhoods where pharmacy owners decided to sell marijuana are probably different from neighborhoods where they decided to pass over the opportunity to sell marijuana (we already know that owners and representatives of the pharmacies differ on their opinions towards the policy). The question is whether the amount of difference between the two groups of pharmacies would mislead our comparison to learn about the effects of selling marijuana on crime victimization and attitudes. What is the standard to judge whether a comparison might mislead us? We know that if an intervention is randomly assigned, then the comparisons based on that intervention would not confuse causal effects with confounding. \citet{hansen2008cbs} develop a formal way to compare a given observed comparison with what would be expected if that comparison had been randomized: in essence, they develop a hypothesis test for the hypothesis that the neighbors of selling pharmacies and the neighbors of not selling pharmacies might have been allocated at random. In our case, the raw comparison of attitudes around pharmacies selling marijuana and those not selling marijuana, confirms our hunch that the places do not in fact compare favorably to a randomized experiment:  the Hansen and Bowers omnibus test across 81 covariates reports a $p$-value of .07. The boxplot, Figure XX, in the  appendix shows that very few of the covariates differed between the two groups of people by more than about .07 standard deviations, although one or two variables at the pharmacy level (average age and number of violent robberies) did differ appreciably at or beyond 1 standard deviation. That figure uses black dots to show when one-by-one hypothesis tests of the hypothesis of no difference between groups indicate a significant departure from the null. We see 11 $p$-values less than .05 out of 81 unadjusted tests (a Holm adjustment would yield only 3 $p$-values different from 1.0, and lowest $p$-value would be .38). In 81 tests of the null of no difference, we would expect approximately 4 tests with $p \le .05$ if there were truly no difference. The excess of 11-4 = 7 tests here is just another way to show what the omnibus $d^2$ tests reported --- more work can be done to improve our baseline comparison.

%\begin{figure}[htbp!]
%\centering
%\caption{Absolute standardized differences of means between the neighbors of the 16 selling pharmacies and the 42 comparison pharmacies} 
%\label{fig:initbal}
%\includegraphics[width=.8\textwidth]{./media/initial_balance_plot.pdf}

%\scriptsize{Although very few differences were larger than 1 sd, approximately 11 differences were inconsistent with random assignment (unadjusted $p \le .05$ (colored with black dots). The omnibus balance test across all 81 covariate relationships produces a $p=.07$ \citep{hansen2008cbs}.}
%\end{figure}

\subsection{Designing a Better Baseline Comparison}
If we had managed to convince the government to issue the registrations by lottery and if a large pool of pharmacies had entered the lottery, we would be able to say that the pharmacies and nearby neighborhoods selected by lottery to sell marijuana would be no different from the pharmacies and associated neighborhoods losing the lottery. This would also be the case if the lottery occurred in groups of places --- say, the right to sell were randomly assigned among pharmacies in Montevideo and also among pharmacies outside of Montevideo. This kind of design would yield a block-randomized experiment. If Montevideo/Interior were the only observed covariate that might confound our comparison, we could directly compare pharmacies to each other within area, and we could compare that stratified design to a hypothetical block-randomized experiment in order to assess the comparison (just as we did above when we compared the simple selling-vs-not selling comparison to a simple randomized experiment with 16 treated and 42 control groups and 81 background covariates). This is what we do below. We use an optimization strategy to produce a series of strata that collectively, across 81 observed covariates, compares favorably to a hypothetical block-randomized experiment. Of course, we do not claim that a stratified design substitutes for an experiment --- we can only speak to differences on 81 covariates, not on all possible observed and unobserved as we could if we had an actually randomized experiment. Later, in \S~\ref{sec:sensitivity}, we assess the extent to which we might change our substantive conclusions based on the stratified design due to the influence of unobserved covariates.\footnote{See \citep[Chapter 3]{rosenbaum2010design} for an overview of this method of sensitivity analysis.}


DESCRIBE THE STRATIFICATION IN SUBSTANTIVE TERMS AND ALSO AS COMPARED TO A HYPOTHETICAL BLOCK-RANDOMIZED EXPERIMENT ON 81 COVARIATES.

\section{Outcome Analysis}
\subsection{Comparisons between Pharmacies}
We run different comparisons considering the distinct status of pharmacies, looking after different effects (see Table \ref{tab:comparisons}). The first comparison is between survey responses of neighbors of all pharmacies registered to sell marijuana in the time period of our study. We named this group as ``sellers'' and includes ``wholetime'', ``dropouts'' and ``newcomers''. We compare ''sellers'' versus responses of neighbors of non-selling pharmacies, conditioning on matched set. If having a selling pharmacy in the neighborhood has any effect on crime victimization and public insecurity perceptions, this effect must be shown comparing these two groups.

A second comparison is between neighbors of pharmacies that sold marijuana at the beginning of the implementation of the policy , which includes ``wholetime'' and ``dropouts'', with neighbors of non-selling pharmacies. It differs from the previous one because excludes the ``newcomers'' pharmacies for which we do not have baseline information, but the expectations about the effect of selling marijuana on outcomes are the same.

The third comparison is done in order to provide robustness to the analysis. We compare ``placebo'' pharmacies with ``sellers'' selling pharmacies. This would help us to dismiss a possible self-selection bias brought by specific characteristics of selling pharmacies. As we mentioned, the ``placebo'' are two pharmacies that established contact with the IRCCA to be a part of the dispensary network but never actually started to sell marijuana. We consider these two cases as placebos. In this comparison, placebos neighbors would be compared with neighbors of ``sellers''. The expectation is to find an effect on ``sellers'' neighbors if the effect is about selling. On the contrary, if the effect is about some characteristic related with the pharmacies' decision to sell, it won't be any difference between the two groups. Another version of this match is the fourth comparison between ``placebo'' and neighbors of non-selling pharmacies. In this case, the expectation is the opposite: no effect if the effect is about selling.

Finally, we will compare pharmacies depending on how long they have been selling  marijuana. The argument is that if there is any effect of selling this would be higher among neighbors of pharmacies that have been selling for more time. In order to test that we compare neighbors of ``wholetime'' pharmacies that were exposed to the selling during the entire period of analysis versus neighbors of pharmacies that, either because they dropped out (``dropouts'') or because they joined later (``newcomers''), were exposed less intensively, named as ``part-time.'' We expect larger effects on the ``wholetime'' than on the ``part-time''. Among the five comparisons, the first two are about the effect of having a marijuana  selling pharmacy closed by, while the other three are robust tests.

\begin{table}[h]
        \centering
        \footnotesize
        \caption{Pharmacies Group Comparisons}
    \label{tab:comparisons}
    \begin{tabular}{clll}
\hline
Comparison    &		&		&	Expected effect	\\ \hline
1	      &	Sellers (wholetime) \&	&	Non-sellers	&	Effect on ``sellers'' if effect is	\\
	      &	dropouts\& newcomers)	&		&	about selling	\\
2	      &	Baseline (16 wholetime)	&	Non-sellers	&	Effect on ``baseline'' if the effect is 	\\
	      &	\& dropouts)	&		&	about selling	\\
3	      &	Sellers (wholetime) \&	&	Placebo	&	Effect if it is about actually selling or 	\\
	      &	dropouts \& newcomers)	&		&  no effect if it is about the decision of selling	\\
4	      &	Placebo 	&	Non-sellers	&	No effect if the effect is about selling. 	\\
5	      &	Wholetime	&	Part-time	&	Larger effect on ``wholetime'' 	\\
	      &		&	 (dropouts \& newcomers)	&	than on the ``part-time''	\\ \hline
\end{tabular}
\end{table}

\section{Sensitivity to Unobserved Confounds} \label{sec:sensitivity}
Our stratified research design successfully adjusted for 81 covariates, but it did not adjust for the many other unobserved differences existing between the selling and not-selling pharmacies. In this section we ask how severe the unobserved differences would have to be before we would change our substantive interpretation of our results. This kind of sensitivity analysis a formalized what-if exercise because, in fact, we do not know how much unobserved confounding to worry about.

TO DO.

\section{Results?}

\section{Discussion and Final Remarks}
This paper has a twofold purpose. On one side, it wants to contribute to the literature which tackles the causal effect between crime and marijuana legalization. On the other side, it aims to evaluate the impact of marijuana regulation in Uruguay on insecurity perceptions and crime victimization, and by doing that inform the public policy.

Our evidence shows no impact of marijuana selling at pharmacies on crime victimization. Following the evidence of \citet{dragone2019crime} and the previous research mentioned above, it was plausible to expect that more dispensaries generate lower crime victimization. It is also truth that, considering the small participation of the marijuana trade in the illegal drug market, the effect of marijuana legalization should be moderate. In Caulkins et al.words: ``Where the problem is already modest, the potential changes are also necessarily modest'' \citep[154]{caulkins2015considering}.

\hl{As we already posed, in the Uruguayan context, where the national crime rates had steadily growth in the last years, we could expect that crime victimization goes upwards both among neighbors of marijuana selling pharmacies and also among others citizens that do not live close by a marijuana dispensary, but it might be a smaller growth among the former. Regarding crime victimization, the preliminary evidence presented in the article does not show that effect.}

\hl{Which are the mechanisms that we believe are behind the link between the legalization of marijuana and crime? More specifically, why would the introduction of legal marijuana dispensaries cause a decrease in crime victimization and the other outcomes associated with public security? The causal mechanisms behind this process might be in line with Dragone's work: moving retail cannabis deals from degraded streets to safe, legal shops most likely played a role as it is posed by Becker and Murphy (2013). Following this argument, if the illegal expenditure points (``bocas'') have less ground to move next to the legal dispensaries, maybe we could be attending to an slipping of the ``bocas'' to neighborhoods where no legal marijuana dispensaries are located. We found no significant difference on the identification of ``bocas'' made by neighbors before and after the pharmacies started to sell marijuana, and neither between neighbors that live close to a selling pharmacy and those that do not. However, selling pharmacies owners and employees state that ``bocas'' have diminished in the neighborhood. We surveyed all the owners of marijuana selling pharmacies before and after they started to sell marijuana. In 2017, 84,6\% said they knew there was a ``boca'' in the neighborhood, and in 2018 that percentage diminished to 46,1\%. Among owners of non- selling pharmacies, owners did not recognize any change (63,3\% in 2017 and 62,9\% in 2018).}

\hl{In the case of insecurity perceptions, estimations do not show any statistically significant effehttps://www.overleaf.com/project/5dc57c0fff0cbd0001059422ct between neighbors, meaning that living closed to a selling pharmacies does not modify insecurity perceptions. Perceived impact of the regulation on public security does not present statistically significant differences neither. People not only feel just as insecure in one group or the other, but also they do not visualize the policy as a potential instrument to improve public insecurity, and this does not change when they are directly in contact with the policy. This finding is interesting considering how this policy emerged, the initial frame that the government used to present the bill, and its specific aims regarding public security. Apparently the link between drugs regulation and crime is not so obvious for treated neighbors either.}

\hl{On the contrary, people who live near a selling pharmacy believe that the regulation is improving the fight against drug trafficking, in contrast with people who does not live near one. Although we do not have much information on the casual mechanisms behind this perception, it might be that regular contact with consumers persuade people to consider that the sale at pharmacies it is actually a way of reducing illegal drug markets, not forcing ordinary people to be in contact with dealers or in the illegal expenditures points. In other words, being in straight contact with the policy improves the perceived impact against the illegal market. An unexpected result is that marijuana selling pharmacies reduce social disorder in the neighborhood. This preliminary finding goes against our expectation and needs further research.}

\hl{As it was stated at the beginning, the purpose of this paper is to make a contribution to the public policy discussion on whether marijuana legalization produce social goods or not. Uruguay was the first country to fully regulate the marijuana market, Canada the second in 2018, and it is probable that more countries would follow the same path. As a result, it is extremely relevant to understand the effects of those legalizations on different social, economic and political outcomes. In particular it looks at a series of outcomes that were key in the Uruguayan legalization policy process: crime and public insecurity perception. The data analysis presented in this version is a first cut. We need to perform a computational matching to analyze the results by clusters of similar pharmacies, in order to compare selling and non-selling pharmacies in a way that comes close to a hypothetical block-randomized experiment, despite we know that there might be always unobserved variables that might have an effect on our outcomes of interest.}


DISCUTIR SOBRE EL TIEMPO DE EXPOSICIÓN AL TRATAMIENTO 
CONTEXTO AL ALZA DEL DELITO
CONTEXTO AL ALZA DE PERCEPCIÓN DE INSEGURIDAD
CAMBIOS EN LAS PERCEPCIONES, OPINIONES
HABLAR DE LA DIRECCIÓN DE LOS EFECTOS AUNQUE NO SEAN SIGNIFICATIVOS

FUTURE STEPS

\printbibliography[title={8 References}]

\newpage
\appendix % is redefined by hyperref and a mess, do the appendix stuff by hand instead
% problems with pdf bookmarks and appendix caused by resetting section counter.
% solution here: http://www.latex-community.org/forum/viewtopic.php?f=45&t=4950
%\begin{singlespacing}
\newcounter{mycounter}
\setcounter{mycounter}{0}\textbf{}
\let\osection\section
\renewcommand{\section}{\stepcounter{mycounter}\osection}
\renewcommand\thesection{Appendix \Alph{mycounter}}

\section{Pharmacies Description}
\begin{table}
    \begin{small}
    \centering
    \caption{Balance test for pharmacies variables in baseline}
    \addtolength{\tabcolsep}{-3pt}
    \label{tab:pharmaciesperceptions}
    \begin{tabular}{lccccc}
&	Mean diff.	&	Std. Error	&	P value	&	[95\% Conf. Interval]\\	\hline
Pharmacies' general characteristics	&		&		&		&	\\	\hline
Number of individual stores 	&	.7268908 	&	.2582723	&	0.0067 **	&	.2097093    1.244072\\
Number of employees per store	&	-.3403361	&	1.080.218	&	 0.7539	&	-2.503435    1.822763\\
Years of functioning	&	3.081232	&	9.527576	&	0.7476	&	-15.99741    22.15988\\
Family business	&	.1539634 	&	 .1357056 	&	0.2615 	&	-.1179967 .4259235	\\  \hline
Pharmacies' infrastructure	&		&		&		&			\\	\hline
Alarm with response	&	.0686275	&	.1740328 	&	 0.6948	&	-.2798672    .4171221	\\
Security guard during the day 	&	-.2338936	&	 .1714943	&	0.1780 &	-.5773049    .1095178	\\
Security cameras 	&	 -.1845238	&	 .1605587	&	0.2553 	&	-.5061616    .1371139	\\
Security guard during the night	&	-.1160714 	&	 .0897231	&	0.2011 &	 -.2958084    .0636655	\\
Security measures at the door	&	.0503049	&	.14968	&	0.7381	&	-.2496605    .3502703	\\ \hline
Insecurity perceptions and crime	&		&		&		&	\\	\hline
Crime victimization 	&	-.1806723	&	.1325175 	&	0.1781	&	 -.446034    .0846895\\
Neighborhood insecurity perception	&	-.0517241	&	.2079464 	&	0.8047	&	-.471088    .3676398\\
Reported existence of ``bocas''	&	.2128205	&	.152596	&	0.1706 	&	-.0953534   	 .5209944	\\
Trust in people of the neighborhood	&	.1402715	&	.151151	&	0.3575	&	-.162768     .443311	\\ \hline
Opinions about the policy	&		&		&		&	\\	\hline
Agreement with marijuana regulation 	&	1.314.024	&	.3639564	&	0.0007**	&	.5846394    2.043409	\\
Agreement with sale at pharmacies	&	1.695.839	&	.2827049	&	0.0000 **	&	 1.129513    2.262165	\\
Perceived impact on public health 	&	.3823529 	&	.1589435 	&	0.0200**	&	.062944    .7017619	\\
Perceived impact on public security 	&	.3482143	&	.1617707	&	0.0363**	&	.0231239    .6733047	\\
Perceived impact on fight against drug-trafficking	&	.1045045	&	.202857	&	0.6087	&	-.3029457    .5119547	\\
Perceived impact on individual liberties	&	.3605769	&	.1763803	&	 0.0459**	&	 .0068029     .714351	\\\hline
Opinions about the substance and its users	&		&		&		&		\\ \hline
Agreement with ``marijuana users	&		&		&		&		\\
are a threat to society''	&	.9803922	&	 .5347	&	0.0719	&	-.0903262     2.05111	\\
Agreement with ``marijuana is a gateway drug''	&	1.535014	&	.6544214	&	0.0225** 	&	.2245578     2.84547	\\
Agreement with ``marijuana use 	&		&		&		&			\\
is detrimental to health''	&	-.4672619	&	.6445555	&	0.4715	&	-1.758462    .8239378	\\	\hline
\end{tabular}
**Significant at 95\%.
\end{small}
\end{table}

\newpage
\newgeometry{margin=1cm}
\begin{landscape}
  \tiny
    \begin{longtable}[htbp]{@{}p{1.2cm}p{1cm}p{1.5cm}p{1cm}p{1cm}p{1cm}p{1cm}p{1cm}p{1cm}p{1cm}p{1cm}p{1.2cm}p{1cm}p{1cm}@{}}
    \caption{Indicators for Selected  Pharmacies}\label{tab:phlongtable}  \tabularnewline
    \toprule
Locality & Pharmacy's name  & Administrative Region (departamento) & Treatment & Assaults per Neighborhood/Locality & Assaults per administrative region & Robberies per Locality & Robberies per administrative region & Homicides per administrative region & Total population of the administrative region & Total population per Locality & Locality's population density &
Average income of the Locality 	&Average age of the Neighborhood/Locality  \tabularnewline
%\midrule
\endhead
Malvin Norte	&	Pitagoras	&	Montevideo	&	Yes	&	871	&	11409	&	184	&	31.137	&	113	&	1.305.082	&	19.916	&	11.620	&	19.302	&	39  \tabularnewline
Cordon	&	Galena	&	Montevideo	&	No	&	1469	&	11409	&	222	&	31.137	&	113	&	1.305.082	&	42.456	&	18.629	&	29.580	&	39	\tabularnewline
Aguada	&	Roosevelt	&	Montevideo	&	No	&	514	&	11409	&	514	&	31.137	&	113	&	1.305.082	&	18.557	&	8.982	&	25.982	&	40	\tabularnewline
Belvedere	&	Belvedere	&	Montevideo	&	No	&	480	&	11409	&	480	&	31.137	&	113	&	1.305.082	&	21.970	&	6.861	&	18.376	&	41	\tabularnewline
Paso Molino	&	Mastil	&	Montevideo	&	No	&	0	&	11409	&	0	&	31.137	&	113	&	1.305.082	&	21.970	&	6.861	&	18.376	&	40	\tabularnewline
Pocitos	&	Brito del Pino	&	Montevideo	&	No	&	1221	&	11409	&	171	&	31.137	&	113	&	1.305.082	&	67.992	&	21.660	&	42.403	&	44	\tabularnewline
Sayago	&	Farmacia Ariel	&	Montevideo	&	No	&	626	&	11409	&	189	&	31.137	&	113	&	1.305.082	&	14.692	&	5.625	&	21.465	&	38	\tabularnewline
Cordon	&	La caja	&	Montevideo	&	No	&	1469	&	11409	&	222	&	31.137	&	113	&	1.305.082	&	42.456	&	18.629	&	29.580	&	38	\tabularnewline
Curva de Maroñas	&	Lulisan	&	Montevideo	&	No	&	0	&	11409	&	239	&	31.137	&	113	&	1.305.082	&	20.812	&	7.133	&	15.591	&	41	\tabularnewline
Parque Rodo	&	Farmashop 50	&	Montevideo	&	No	&	429	&	11409	&	0	&	31.137	&	113	&	1.305.082	&	12.944	&	16.898	&	33.781	&	41	\tabularnewline
Ciudad Vieja	&	Cielmar	&	Montevideo	&	No	&	593	&	11409	&	0	&	31.137	&	113	&	1.305.082	&	12.555	&	5.947	&	23.112	&	41	\tabularnewline
Union	&	Milena	&	Montevideo	&	No	&	1582	&	11409	&	299	&	31.137	&	113	&	1.305.082	&	39.880	&	9.975	&	21.562	&	43	\tabularnewline
Buceo	&	Farmashop 58	&	Montevideo	&	No	&	1155	&	11409	&	251	&	31.137	&	113	&	1.305.082	&	36.998	&	8.905	&	27.440	&	43	\tabularnewline
Brazo oriental	&	Farmacia Goñi Central	&	Montevideo	&	No	&	456	&	11409	&	0	&	31.137	&	113	&	1.305.082	&	16.812	&	8.976	&	21.519	&	43	\tabularnewline
Sayago	&	Sangar	&	Montevideo	&	No	&	626	&	11409	&	189	&	31.137	&	113	&	1.305.082	&	14.692	&	5.625	&	21.465	&	43	\tabularnewline
La Blanqueada	&	Guarani	&	Montevideo	&	No	&	0	&	11409	&	0	&	31.137	&	113	&	1.305.082	&	9.600	&	12.245	&	31.489	&	43	\tabularnewline
Parque Batlle	&	FARMASHOP 52	&	Montevideo	&	No	&	1049	&	11409	&	185	&	31.137	&	113	&	1.305.082	&	31.153	&	9.231	&	36.782	&	40	\tabularnewline
Cordon	&	Pigalle	&	Montevideo	&	No	&	1469	&	11409	&	222	&	31.137	&	113	&	1.305.082	&	42.456	&	18.629	&	29.580	&	43	\tabularnewline
Malvin	&	El tunel	&	Montevideo	&	No	&	871	&	11409	&	217	&	31.137	&	113	&	1.305.082	&	28.102	&	8.027	&	37.732	&	40	\tabularnewline
Malvin	&	San Roque	&	Montevideo	&	No	&	871	&	11409	&	217	&	31.137	&	113	&	1.305.082	&	28.102	&	8.027	&	37.732	&	40	\tabularnewline
Centro	&	Antartida	&	Montevideo	&	Yes	&	1120	&	11409	&	173	&	31.137	&	113	&	1.305.082	&	22.120	&	17.055	&	34.049	&	44	\tabularnewline
Pocitos	&	CACERES	&	Montevideo	&	Yes	&	1221	&	11409	&	171	&	31.137	&	113	&	1.305.082	&	67.992	&	21.660	&	42.403	&	44	\tabularnewline
Ciudad vieja	&	Tapie	&	Montevideo	&	Yes	&	593	&	11409	&	0	&	31.137	&	113	&	1.305.082	&	12.555	&	5.947	&	23.112	&	43	\tabularnewline
Aguada	&	Sildia	&	Montevideo	&	No	&	514	&	11409	&	514	&	31.137	&	113	&	1.305.082	&	18.557	&	8.982	&	25.982	&	43	\tabularnewline
La Blanqueada	&	Quintela	&	Montevideo	&	No	&	0	&	11409	&	0	&	31.137	&	113	&	1.305.082	&	9.600	&	12.245	&	31.489	&	41	\tabularnewline
Ciudad de Artigas	&	Saga	&	Artigas	&	Yes	&	14	&	25	&	386	&	1.647	&	0	&	73.377	&	40.658	&	2.740	&	11.733	&	36	\tabularnewline
Bella Unión	&	Santa Cecilia Centro	&	Artigas	&	No	&	7	&	25	&	2	&	1.647	&	0	&	73.377	&	40.658	&	2.740	&	11.733	&	36	\tabularnewline
Ciudad de Artigas	&	Horandre	&	Artigas	&	No	&	14	&	25	&	386	&	1.647	&	0	&	73.377	&	40.658	&	2.740	&	11.733	&	36	\tabularnewline
Las Toscas	&	Las toscas	&	Canelones	&	Yes	&	38	&	2165	&	1160	&	11.490	&	26	&	520.173	&	3.146	&	1.022	&		&		\tabularnewline
Ciudad de Canelones	&	miguel	&	Canelones	&	Yes	&	57	&	2165	&	706	&	11.490	&	26	&	520.173	&	19.865	&	1.582	&	19.130	&	40	\tabularnewline
Parque del Plata	&	bologna	&	Canelones	&	No	&	38	&	2165	&	1160	&	11.490	&	26	&	520.173	&	7.896	&	945	&		&		\tabularnewline
Ciudad de Canelones	&	Pirujas	&	Canelones	&	No	&	57	&	2165	&	706	&	11.490	&	26	&	520.173	&	19.865	&	1.582	&	19.130	&	40	\tabularnewline
Pando	&	Farmacia central	&	Canelones	&	No	&	169	&	2165	&	1045	&	11.490	&	26	&	520.173	&	25.947	&	2.471	&		&		\tabularnewline
Carmelo	&	Carmelo	&	Colonia	&	Yes	&	2	&	20	&	434	&	1.971	&	3	&	123.203	&	18.041	&	1.512	&	15.117	&	38	\tabularnewline
Nueva Palmira	&	Arrieta	&	Colonia	&	No	&	4	&	20	&	100	&	1.971	&	3	&	123.203	&	9.857	&	528	&	15.947	&	38	\tabularnewline
Carmelo	&	Ferrer	&	Colonia	&	No	&	2	&	20	&	434	&	1.971	&	3	&	123.203	&	18.041	&	1.512	&	15.117	&	38	\tabularnewline
Trinidad	&	Nueva Brun	&	Flores	&	Yes	&	15	&	7	&	644	&	396	&	0	&	25.050	&	21.429	&	3.221	&	15.694	&	40	\tabularnewline
Trinidad	&	Osta	&	Flores	&	No	&	15	&	7	&	644	&	396	&	0	&	25.050	&	21.429	&	3.221	&	15.694	&	40	\tabularnewline
Ismael Cortinas	&	Vidal	&	Flores	&	No	&	0	&	7	&	12	&	396	&	0	&	25.050	&	918	&	963	&		&		\tabularnewline
Minas	&	Gortari	&	Lavalleja	&	Yes	&	27	&	17	&	647	&	1.009	&	5	&	58.815	&	38.446	&	2.135	&	15.879	&	37	\tabularnewline
Minas	&	Idamar	&	Lavalleja	&	No	&	27	&	17	&	647	&	1.009	&	5	&	58.815	&	38.446	&	2.135	&	15.879	&	37	\tabularnewline
Minas	&	Williman 2	&	Lavalleja	&	No	&	27	&	17	&	647	&	1.009	&	5	&	58.815	&	38.446	&	2.135	&	15.879	&	37	\tabularnewline
Las Flores	&	La Cabina	&	Maldonado	&	Yes	&	0	&	212	&	102	&	6.238	&	12	&	164.298	&	241	&	229	&		&		\tabularnewline
San Carlos	&	Alvariza	&	Maldonado	&	No	&	43	&	212	&	755	&	6.238	&	12	&	164.298	&	27.471	&	3.656	&	16.948	&	38	\tabularnewline
Maldonado	&	Maldonado	&	Maldonado	&	No	&	31	&	212	&	1106	&	6.238	&	12	&	164.298	&	62.590	&	4.916	&	16.127	&	36	\tabularnewline
Ciudad de Paysandú	&	Medicci	&	Paysandú	&	Yes	&	38	&	74	&	834	&	2.635	&	4	&	113.107	&	76.412	&	3.539	&	15.082	&	38	\tabularnewline
Villa Quebracho	&	Guaviyu	&	Paysandú	&	No	&	0	&	74	&	33	&	2.635	&	4	&	113.107	&	2.853	&	2.202	&	.	&	.	\tabularnewline
Termas de Guaviyú	&	Termal guaviyu	&	Paysandú	&	Yes	&	0	&	74	&	19	&	2.635	&	4	&	113.107	&	38	&	44	&	.	&	.	\tabularnewline
Guichón	&	Lombardi/Guichón	&	Paysandú	&	No	&	1	&	74	&	19	&	2.635	&	4	&	113.107	&	5.039	&	1.577	&	10.181	&	37	\tabularnewline
Ciudad de Paysandú	&	San Roque	&	Paysandú	&	No	&	29	&	74	&	534	&	2.635	&	4	&	113.107	&	76.412	&	3.539	&	15.082	&	38	\tabularnewline
Ciudad de Paysandú	&	Dorotte II	&	Paysandú	&	No	&	29	&	74	&	534	&	2.635	&	4	&	113.107	&	76.412	&	3.539	&	15.082	&	38	\tabularnewline
Dayman	&	Albisu Termal	&	Salto	&	Yes	&	17	&	122	&	1187	&	2.978	&	7	&	124.861	&	356	&	209	&		&		\tabularnewline
Ciudad de Salto	&	Farmacia Pasteur	&	Salto	&	No	&	16	&	122	&	804	&	2.978	&	7	&	124.861	&	104.011	&	2.812	&	16.209	&	38	\tabularnewline
Ciudad de Salto	&	Nueva Republica	&	Salto	&	No	&	16	&	122	&	804	&	2.978	&	7	&	124.861	&	104.011	&	2.812	&	16.209	&	38	\tabularnewline
Libertad	&	Bedegain	&	San José	&	Yes	&	10	&	143	&	440	&	1.815	&	2	&	108.304	&	10.167	&	1.764	&	15.757	&	37	\tabularnewline
Ciudad del Plata	&	Del 26	&	San José	&	No	&	109	&	143	&	507	&	1.815	&	2	&	108.304	&	31.146	&	1.200	&	21.655	&	37	\tabularnewline
San José de Mayo	&	Bellini	&	San José	&	No	&	20	&	143	&	646	&	1.815	&	2	&	108.304	&	36.743	&	2.641	&	18.934	&	40	\tabularnewline
Paso de los Toros	&	Bengoechea	&	Tacurembó	&	Yes	&	1	&	16	&	93	&	907	&	6	&	90.051	&	12.985	&	1.341	&	12.768	&	36	\tabularnewline
Paso de los Toros	&	Demilton	&	Tacurembó	&	No	&	1	&	16	&	93	&	907	&	6	&	90.051	&	12.985	&	1.341	&	12.768	&	36	\tabularnewline
Tacuarembó	&	Dini	&	Tacurembó	&	No	&	19	&	15	&	600	&	907	&	6	&	90.051	&	54.757	&	1.721	&	13.940	&	37	\tabularnewline
Pocitos	&	Camaño	&	Montevideo	&	Yes	&	1221	&	11409	&	171	&	31.137	&	113	&	1.305.082	&	67.992	&	21.660	&	42.403	&	44	\tabularnewline
Brazo oriental	&	Silleda	&	Montevideo	&	Yes	&	456	&	11409	&	0	&	31.137	&	113	&	1.305.082	&	16.812	&	8.976	&	21.519	&	43	\tabularnewline
Flor de Maroñas	&	Constitución Sur	&	Montevideo	&	Yes	&	0	&	11409	&	239	&	31.137	&	113	&	1.305.082	&	20.812	&	7.133	&	15.591	&	42	\tabularnewline
Punta Carretas	&	Lilen	&	Montevideo	&	Yes	&	617	&	11409	&	170	&	31.137	&	113	&	1.305.082	&	24.181	&	8.858	&	46.759	&	43	\tabularnewline
\bottomrule
\end{longtable}

Source: Own elaboration. Population data from National Census (2011). Income and age data from National Household Survey (2017). Criminality data per administrative regions from Ministry of Interior (1st. semester 2018) Criminality data per neighborhoods in Montevideo from Ministry of Interior (1st. semester 2018).  Criminality data per localities in the rest of the country from Ministry of Interior (2016).
\end{landscape}
\restoregeometry


\section{Outcome Variable Description}
\begin{tiny}
\begin{tabular}{lllcc}
\textbf{Variable }	&	\textbf{Survey question} 	&	\textbf{Scale}	&	\textbf{Min.}	&	\textbf{Max.}		\\	\hline
Country insecurity  perception	&	In general, in your country, do you feel very safe	&	1. Very unsafe 	&	1	&	4	\\
	&	somewhat unsafe or very unsafe?	&	2. Somewhat unsafe 	&		&		\\
	&		&	3. Somewhat safe 	&		&		\\
	&		&	4. Very safe 	&		&		\\\hline
Neighborhood 	&	And in the neighborhood where you live, 	&	1. Very unsafe 	&	1	&	4	\\
insecurity perception	&	do you feel very safe, somewhat safe,	&	2. Somewhat unsafe 	&		&		\\
	&	somewhat unsafe or very unsafe?	&	3. Somewhat safe 	&		&		\\
	&		&	4. Very safe 	&		&		\\\hline
Crime victimization in the last 	&	Now, changing the subject, have you been a victim of 	&	0. Yes	&	0	&	1	\\
12 months	&	any type of crime in the past 12 months? That is, have	&	1. No 	&		&		\\
	&	you been a victim of robbery, burglary, assault, fraud,	&		&		&		\\
	&	blackmail, extortion, violent threats or any other type	&		&		&		\\
	&	of crime in the past 12 months?	&		&		&		\\\hline
Perceived impact on public security	&	Now regarding to public safety, because of this law,	&	1.  Worst (is already worst)	&	1	&	3	\\
	&	 do you think the country will be better, will 	&	2. Same	&		&		\\
	&	remain the same or will be worst?	&	3. Better (is already better)	&		&		\\\hline
Perceived impact on drug trafficking	&	Now regarding to drug trafficking, because of this law,	&	1.  Worst (is already worst)	&	1	&	3	\\
	&	 do you think the country will be better, will 	&	2. Same	&		&		\\
	&	remain the same or will be worst?	&	3. Better (is already better)	&		&		\\\hline
Reported existence of ``bocas'' &	Based on what you know or hear, there is a any ``bocas'' &	0. Yes	&	0	&	1	\\
	&	in this area?	&	1. No 	&		&		\\\hline
Social disorder index	&	Built with the folllowing questions:	&	1. Very much	&	1	&	4	\\
	&		&	2. Somewhat	&		&		\\
	&		&	3. Llittle	&		&		\\
	&		&	4. Not at all	&		&		\\
Presence of young people or children 	&	Observational	&	1. Very much	&	1	&	4	\\
in the streets without doing	&		&	2. Somewhat	&		&		\\
 anything, who are wandering	&		&	3. Llittle	&		&		\\
	&		&	4. Not at all	&		&		\\
Presence of   people drunk or 	&	Observational	&	1. Very much	&	1	&	4	\\
stoned in the streets	&		&	2. Somewhat	&		&		\\
	&		&	3. Llittle	&		&		\\
	&		&	4. Not at all	&		&		\\
Presence of people discussing in 	&	Observational	&	1. Very much	&	1	&	4	\\
aggressive or violent (speaking in 	&		&	2. Somewhat	&		&		\\
a tone of voice very high, 	&		&	3. Llittle	&		&		\\
with anger)	&		&	4. Not at all	&		&		\\\hline
Citizens insertion in 	&	Built with the folllowing questions:	&		&		&		\\
neighborhood*	&		&	-	&	0	&	1	\\
Talk with your neighbors	&	Finally, thinking about the activities you do in 	&	1. Never	&		&		\\
	&	this neighborhood, please tell me how many times you:	&	2. Once or twice a year	&	1	&	4	\\
	&	talk with your neighbors	&	3. Once or twice a month	&		&		\\
	&		&	4. Once a week	&		&		\\
Organize meetings with the 	&	Finally, thinking about the activities you do in 	&	1. Never	&		&		\\
neighbors to improve 	&	this neighborhood, please tell me how many times you:	&	2. Once or twice a year	&	1	&	4	\\
the neighborhood	&	organize meetings with the 	&	3. Once or twice a month	&		&		\\
	&	neighbors to improve the neighborhood	&	4. Once a week	&		&		\\
Use educational services 	&	Finally, thinking about the activities you do in 	&	1. Never	&	1	&	4	\\
(kindergarten, school, high school)	&	this neighborhood, please tell me how many times you:	&	2. Once or twice a year	&		&		\\
 in the neighborhood	&	use educational services (kindergarten, school, 	&	3. Once or twice a month	&		&		\\
	&	high school) in the neighborhood	&	4. Once a week	&		&		\\
Use health services 	&	Finally, thinking about the activities you do in 	&	1. Never	&		&		\\
(doctor, hospital) of the	&	this neighborhood, please tell me how many times you:	&	2. Once or twice a year	&	1	&	4	\\
neighborhood	&	use health services (doctor, hospital) of the	&	3. Once or twice a month	&		&		\\
	&	neighborhood	&	4. Once a week	&		&		\\
Buy at the shops in the 	&	Finally, thinking about the activities you do in 	&	1. Never	&		&		\\
neighborhood	&	this neighborhood, please tell me how many times you:	&	2. Once or twice a year	&	1	&	4	\\
	&	buy at the shops in the neighborhood	&	3. Once or twice a month	&		&		\\
	&		&	4. Once a week	&		&		\\
Do recreational activities 	&	Finally, thinking about the activities you do in 	&	1. Never	&		&		\\
in the neighborhood	&	this neighborhood, please tell me how many times you:	&	2. Once or twice a year	&	1	&	4	\\
	&	do recreational activities in the neighborhood	&	3. Once or twice a week	&		&		\\
	&		&	4. Once a week	&		&		\\\hline
*Standarized
\end{tabular}
\end{tiny}
\restoregeometry


\newpage The social disorder index was built using three variables: presence of young people loitering, presence of drunk or stoned people in the streets, and presence of people arguing with each other. Each of these variables have the following values: 1 (Very much), 2 (Somewhat), 3 (Little), and 4 (Not at all). In order to construct the index we assume intermediate substitutability among variables and assign the mean value \citep{goertz2006social}.

Citizens' Insertion on their Neighborhood is an index constructed with eight variables grouped in four dimensions. The dimensions are: use of services (education and health) in the neighborhood, contact among neighbors (chat and/or meet for collective action activities), perform recreational activities in the neighborhood, and shopping in the neighborhood. We assume intermediate substitutability among dimensions and assign the mean value. The use of services dimension has two indicators: use of educational services (kindergarten, school,high school) in the neighborhood and use of health services (doctor, hospital) in the neighborhood. We assume total substitutability among the two indicators and assign the maximum value  \citep{goertz2006social}.Contact among neighbors dimension also has two indicators: talk with your neighbors and meets/ organizes with the neighbors for any improvement activity for the neighborhood. We also assume total substitutability among the two indicators and assign the maximum value. Perform recreational activities in the neighborhood is measured by the question: do you perform any recreational activities in the neighborhood? Finally, shopping in the neighborhood is measured by the question: do you purchase in warehouses or stores in the neighborhood? Each indicator is measured using the same scale:1 (Never), 2 (Once or twice a year), 3 (Once or twice a month), and 4 (Once a week).  Citizens' insertion is an index standardized, values go from 0 to 1.

\section{Details of the search for stratification}
Here we will describe in more detail how we used a full optimal matching algorithm \citep{hansen:2004} to find a stratified comparison between selling and non-selling pharmacies that met the standard of comparing favorably with a hypothetical block-randomized experiment.

\section{Threats to Inference: Attrition and Spillovers}
One problem for this design could be sample attrition. It is possible that some people moved, leaving either the treatment or the control group. In general, moving is not a frequent behavior in Uruguay. Although we do not know for sure how many respondents from the baseline survey moved before the second round, we do know that only 4.33 percent (n=30) of those interviewed in the second round moved after we run the baseline.

We minimize spillover between pharmacies by ensuring that the comparison pharmacies were at least 6 blocks away from the selling pharmacies.
\end{document}
