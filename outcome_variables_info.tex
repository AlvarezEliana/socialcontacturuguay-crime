
\section{Outcome Variables}\label{sec:outcome-variables}

\begin{table}[htbp]
\caption{Outcome Variable Description}\label{tab:outcome_variables_info}
\begin{tiny}
\begin{tabular}{lllcc}
\textbf{Variable }	&	\textbf{Survey question} 	&	\textbf{Scale}	&	\textbf{Min.}	&	\textbf{Max.}		\\	\hline
Country insecurity  perception	&	In general, in your country, do you feel very safe	&	1. Very unsafe 	&	1	&	4	\\
	&	somewhat unsafe or very unsafe?	&	2. Somewhat unsafe 	&		&		\\
	&		&	3. Somewhat safe 	&		&		\\
	&		&	4. Very safe 	&		&		\\\hline
Neighborhood 	&	And in the neighborhood where you live, 	&	1. Very unsafe 	&	1	&	4	\\
insecurity perception	&	do you feel very safe, somewhat safe,	&	2. Somewhat unsafe 	&		&		\\
	&	somewhat unsafe or very unsafe?	&	3. Somewhat safe 	&		&		\\
	&		&	4. Very safe 	&		&		\\\hline
Crime victimization in the last 	&	Now, changing the subject, have you been a victim of 	&	0. Yes	&	0	&	1	\\
12 months	&	any type of crime in the past 12 months? That is, have	&	1. No 	&		&		\\
	&	you been a victim of robbery, burglary, assault, fraud,	&		&		&		\\
	&	blackmail, extortion, violent threats or any other type	&		&		&		\\
	&	of crime in the past 12 months?	&		&		&		\\\hline
Perceived impact on public security	&	Now regarding to public safety, because of this law,	&	1.  Worst (is already worst)	&	1	&	3	\\
	&	 do you think the country will be better, will 	&	2. Same	&		&		\\
	&	remain the same or will be worst?	&	3. Better (is already better)	&		&		\\\hline
Perceived impact on drug trafficking	&	Now regarding to drug trafficking, because of this law,	&	1.  Worst (is already worst)	&	1	&	3	\\
	&	 do you think the country will be better, will 	&	2. Same	&		&		\\
	&	remain the same or will be worst?	&	3. Better (is already better)	&		&		\\\hline
Reported existence of ``bocas'' &	Based on what you know or hear, there is a any ``bocas'' &	0. Yes	&	0	&	1	\\
	&	in this area?	&	1. No 	&		&		\\\hline
Social disorder index	&	Built with the folllowing questions:	&	1. Very much	&	1	&	4	\\
	&		&	2. Somewhat	&		&		\\
	&		&	3. Llittle	&		&		\\
	&		&	4. Not at all	&		&		\\
Presence of young people or children 	&	Observational	&	1. Very much	&	1	&	4	\\
in the streets without doing	&		&	2. Somewhat	&		&		\\
 anything, who are wandering	&		&	3. Llittle	&		&		\\
	&		&	4. Not at all	&		&		\\
Presence of   people drunk or 	&	Observational	&	1. Very much	&	1	&	4	\\
stoned in the streets	&		&	2. Somewhat	&		&		\\
	&		&	3. Llittle	&		&		\\
	&		&	4. Not at all	&		&		\\
Presence of people discussing in 	&	Observational	&	1. Very much	&	1	&	4	\\
aggressive or violent (speaking in 	&		&	2. Somewhat	&		&		\\
a tone of voice very high, 	&		&	3. Llittle	&		&		\\
with anger)	&		&	4. Not at all	&		&		\\\hline
Citizens insertion in 	&	Built with the folllowing questions:	&		&		&		\\
neighborhood*	&		&	-	&	0	&	1	\\
Talk with your neighbors	&	Finally, thinking about the activities you do in 	&	1. Never	&		&		\\
	&	this neighborhood, please tell me how many times you:	&	2. Once or twice a year	&	1	&	4	\\
	&	talk with your neighbors	&	3. Once or twice a month	&		&		\\
	&		&	4. Once a week	&		&		\\
Organize meetings with the 	&	Finally, thinking about the activities you do in 	&	1. Never	&		&		\\
neighbors to improve 	&	this neighborhood, please tell me how many times you:	&	2. Once or twice a year	&	1	&	4	\\
the neighborhood	&	organize meetings with the 	&	3. Once or twice a month	&		&		\\
	&	neighbors to improve the neighborhood	&	4. Once a week	&		&		\\
Use educational services 	&	Finally, thinking about the activities you do in 	&	1. Never	&	1	&	4	\\
(kindergarten, school, high school)	&	this neighborhood, please tell me how many times you:	&	2. Once or twice a year	&		&		\\
 in the neighborhood	&	use educational services (kindergarten, school, 	&	3. Once or twice a month	&		&		\\
	&	high school) in the neighborhood	&	4. Once a week	&		&		\\
Use health services 	&	Finally, thinking about the activities you do in 	&	1. Never	&		&		\\
(doctor, hospital) of the	&	this neighborhood, please tell me how many times you:	&	2. Once or twice a year	&	1	&	4	\\
neighborhood	&	use health services (doctor, hospital) of the	&	3. Once or twice a month	&		&		\\
	&	neighborhood	&	4. Once a week	&		&		\\
Buy at the shops in the 	&	Finally, thinking about the activities you do in 	&	1. Never	&		&		\\
neighborhood	&	this neighborhood, please tell me how many times you:	&	2. Once or twice a year	&	1	&	4	\\
	&	buy at the shops in the neighborhood	&	3. Once or twice a month	&		&		\\
	&		&	4. Once a week	&		&		\\
Do recreational activities 	&	Finally, thinking about the activities you do in 	&	1. Never	&		&		\\
in the neighborhood	&	this neighborhood, please tell me how many times you:	&	2. Once or twice a year	&	1	&	4	\\
	&	do recreational activities in the neighborhood	&	3. Once or twice a week	&		&		\\
	&		&	4. Once a week	&		&		\\\hline
*Standarized
\end{tabular}
\end{tiny}
\end{table}
%\restoregeometry

\newpage

The social disorder index was built using three variables: presence of young
people loitering, presence of drunk or stoned people in the streets, and
presence of people arguing with each other. Each of these variables have the
following values: 1 (Very much), 2 (Somewhat), 3 (Little), and 4 (Not at all).
In order to construct the index we assume intermediate substitutability among
variables and assign the mean value \cite{goertz2006social}.

itizens' Insertion on their Neighborhood is an index constructed with eight variables grouped in four dimensions. The dimensions are: use of services (education and health) in the neighborhood, contact among neighbors (chat and/or meet for collective action activities), perform recreational activities in the neighborhood, and shopping in the neighborhood. We assume intermediate substitutability among dimensions and assign the mean value. The use of services dimension has two indicators: use of educational services (kindergarten, school,high school) in the neighborhood and use of health services (doctor, hospital) in the neighborhood. We assume total substitutability among the two indicators and assign the maximum value  \cite{goertz2006social}.Contact among neighbors dimension also has two indicators: talk with your neighbors and meets/ organizes with the neighbors for any improvement activity for the neighborhood. We also assume total substitutability among the two indicators and assign the maximum value. Perform recreational activities in the neighborhood is measured by the question: do you perform any recreational activities in the neighborhood? Finally, shopping in the neighborhood is measured by the question: do you purchase in warehouses or stores in the neighborhood? Each indicator is measured using the same scale:1 (Never), 2 (Once or twice a year), 3 (Once or twice a month), and 4 (Once a week).  Citizens' insertion is an index standardized, values go from 0 to 1.

